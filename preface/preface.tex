\chapter*{Preface}
Work began in earnest on Oregon's Transportation and Land Use Model Integration Program (TLUMIP) almost two decades ago. First generation urban and statewide models were developed as a proof of concept early on, followed by the design and development of a second generation platform. The latter aimed to extend the state of the art in integrated modeling by adopting a dynamic disequilibrium approach. This was inspired by the agent-oriented TRANSIMS project \citep{smith95}, parallel advances in activity-based travel models, and a general dissatisfaction with the aggregate land use-transportation models in use at the time. The initial design called for agent-oriented microsimulations of socioeconomic activity, location choice, and person and commercial travel, as well as the interactions between them. Each part of the system was intended to operate at the level of temporal, spatial, and behavioral resolution most appropriate to it.

The third major thrust of TLUMIP has been its implementation at the Oregon DOT and application to policy, invesment, and project assessments. They three phases were not sequential, and especially not the second and third parts. The Bridge Limitations Study, conducted in 2003-04 using the first generation model, was a defining accomplishment in the program's history, and several others since then have shaped TLUMIP in important ways. 

All three parts of TLUMIP are important stories to be told and learned from. This report describes only the second part --- data and model development --- in detail. The resulting forecasting system is the Statewide Integrated Model (SWIM), Version 2.5 of which is described in this report. Some parts of it have changed little from the original specification, while other parts have evolved considerably. Perhaps none changed as much, or so influenced progress on the work, as the decision from shift away from the microsimulated production allocation framework into what is now the standalone Production-Allocation-Consumption Allocation System (PECAS), described in Chapter \ref{sec:aa-module-chapter}. The resulting SWIM v2.5 system is a better platform today for all of the changes described in the individual chapters.

A lot of blood, sweat, and tears went into the development of the SWIM system. The project benefited immensely from continued support from Oregon DOT leadership over time. Support from FHWA was hugely instrumental in the program's success through their financial support. The vision and dedication of ODOT staff over time has been key to the success of this program, developing, testing and applying this long range planning tool. A number of others have made major contributions to the work described in this report:

\begin{description}\itemsep-3.5pt
\item \hspace{0.4in}John Abraham (HBA Specto)
\item \hspace{0.4in}Carl Batten (ECONorthwest)
\item \hspace{0.4in}Alex Bettinardi (Oregon DOT)
\item \hspace{0.4in}Patrick Costinett (Parsons Brinckerhoff, now retired)
\item \hspace{0.4in}Chris Frazier (Parsons Brinckerhoff, now with Sandia National Laboratories) 
\item \hspace{0.4in}Joel Freedman (Resource Systems Group)
\item \hspace{0.4in}Brian Gregor (Oregon DOT, now with Oregon Systems Analytics)
\item \hspace{0.4in}Jim Hicks (Parsons Brinckerhoff)
\item \hspace{0.4in}Graham Hill (HBA Specto)
\item \hspace{0.4in}John Douglas Hunt (University of Calgary and HBA Specto)
\item \hspace{0.4in}Gregory Macfarlane (Parsons Brinckerhoff)
\item \hspace{0.4in}Yegor Malinovskiy (Parsons Brinckerhoff, now with INRIX)
\item \hspace{0.4in}Ben Stabler (Resource Systems Group)
\item \hspace{0.4in}Erin Wardell (Parsons Brinckerhoff, now with Washington County)
\item \hspace{0.4in}Tara Weidner (Parsons Brinckerhoff, now with Oregon DOT)
\item \hspace{0.4in}Michal Wert (MW Consulting)
\item \hspace{0.4in}Christi Willison (Parsons Brinckerhoff)
\end{description}

An independent peer review panel also heavily influenced the direction of TLUMIP and the evolution of SWIM. The value of their contributions cannot be overstated:

\begin{description}\itemsep-3.5pt
\item \hspace{0.4in}Julie Dunbar (Dunbar Transportation Consulting)
\item \hspace{0.4in}Kimberly Fisher (Transportation Research Board, now at University of Maryland)
\item \hspace{0.4in}Frank Koppelman (Northwestern University)
\item \hspace{0.4in}Keith Lawton (Portland Metro, now retired independent consultant)
\item \hspace{0.4in}Gordon Shunk (Texas Transportation Institute, now deceased)
\item \hspace{0.4in}David Simmonds, David Simmonds Consultancy
\item \hspace{0.4in}Bill Upton (Joined after retiring from Oregon DOT)
\item \hspace{0.4in}Michael Wegener (University of Dortmund and Spiekermann \& Wegener)
\end{description}

It has been an incredible honor and privilege to travel on the TLUMIP journey with all of them. 

This report most likely represents the end of TLUMIP. That is not to say that statewide modeling in Oregon is waning, or that SWIM will not transform modeling at the Oregon DOT. Both appear to be well-established at this point. Perhaps more importantly, looking at problems from the different perspective imposed by the integrated land use-transportation modeling paradigm will enable the agency to quickly expand their focus to pressing new issues. Closing the last chapter of TLUMIP in order for it to become a cornerstone of planning analytics in Oregon feels like its most important contribution.

{
\vspace{14pt}
\setlength\parindent{4.5in}
Rick Donnelly

February 2017
}
